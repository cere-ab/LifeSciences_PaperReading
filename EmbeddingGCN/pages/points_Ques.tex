%% The positive takeaways from the paper
\section{Positive Points}
\begin{itemize}
    \item One of the most straight-forward application of Graph Neural Networks
        for the task of gene prioritization
    \item For the most part, the paper is well described
\end{itemize}


%% The negative takeaways from the paper
\section{Negative Points}
\begin{itemize}
    \item The construction of the combined adjacency matrix (Eq. 1 in the paper)
        follows some logic, however, it is not revealed in the paper
    \item The candidate gene list is selected from the decoding module, but, the
        sorting is done based on the DeepWalk scores. Authors do not provide a
        justification.
    \item The authors have mentioned that they used \textbf{dropout on Adjacency matrix}
        as a means to overcome over-fitting of the GCN. This approach doesn't seem
        right as random edge dropping breaks the domain context.
\end{itemize}


%% The questions that I have of the authors
\section{Questions}
\begin{enumerate}
    \item The cluster loss uses a threshold; apparently it is based on distance
        but, how is it selected?
    \item How is the threshold during gene prioritization selected?
\end{enumerate}