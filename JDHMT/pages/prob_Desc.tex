\begin{sloppypar*}
    In this paper, the authors have attempted to achieve the following three
    objectives:
    \begin{enumerate}
        \item Prepare a format for the representation of the data (called \textit{triples}).
            In essence, it denotes the dimensions for the initial tensors
        \item Whereas the format above is used for storing and manipulating the
            knowledge graph, the actual information was curated from $7$ mainstream
            datasets. A large-scale heterogeneous \textit{gene-disease} network
            was thus constructed \cite{heteroData}
        \item Embeddings for genes and diseases were generated using a \textbf{J}oint
            \textbf{D}ecomposition of \textbf{H}eterogeneous \textbf{M}atrix
            and \textbf{T}ensor optmisation model. The \textit{principal components} 
            thus obtained form the embeddings
    \end{enumerate}
    
    \noindent Further, to evaluate the usefulness of the generated embeddings, two
    approaches were undertaken - \textit{intrinsic}, where, goodness was assessed
    visually through t-SNE plots and \textit{extrinsic}, where, goodness was assessed
    by utilising the embeddings in some downstream task.
\end{sloppypar*}