\section{Jargon Glossary}
		
    \begin{mybox}
        \textbf{functional annotation/description} - description of a function of the protein that a gene produces
    \end{mybox}
    \begin{mybox}
        \textbf{gene} - molecular gene is a sequence of nucleotides in DNA
    \end{mybox}
    \begin{mybox}
        \textbf{gene-expression} - using information from a gene to synthesize either proteins or non-coding RNA
    \end{mybox}
    \begin{mybox}
        \textbf{gene co-expression} - when expression of two or more genes are correlated
    \end{mybox}
    \begin{mybox}
        \textbf{genome} - umbrella term, includes \textit{all} genetic information of an organism
    \end{mybox}
    \begin{mybox}
        \textbf{pathways} - when different genes work in different sequential steps of a biological process, it is called a \textit{genetic pathway}
    \end{mybox}
    \begin{mybox}
        \textbf{transcript} - A primary transcript is the single-stranded ribonucleic acid (RNA) product synthesized by transcription of DNA, and processed to yield various mature RNA products such as mRNAs, tRNAs, and rRNAs
    \end{mybox}
    \begin{mybox}
        \textbf{transcription} - Both DNA and RNA are nucleic acids, which use base pairs of nucleotides as a complementary language. During transcription, a DNA sequence is read by an RNA polymerase, which produces a complementary, antiparallel RNA strand called a primary transcript
    \end{mybox}
    \begin{mybox}
        \textbf{transcriptome}  - \textit{set} of all transcripts (including coding and non-coding)
    \end{mybox}