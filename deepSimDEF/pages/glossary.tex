\begin{mybox}
    \textbf{Biological Process (BP)} - A biological macro-function such as \textit{DNA repair}, which is achieved via multiple molecular activities. \cite{GOdoc}
\end{mybox}

\begin{mybox}
    \textbf{Cellular Component (CC)} - The location occupied by a macro-molecular machine, relative to the cellular compartments and structures. \cite{GOdoc}
\end{mybox}

\begin{mybox}
    \textbf{Functional Similarity (FS) (\textit{of genes})} - A quantitative measure to enable comparison of genes for their roles in biological processes and molecular functions. Most measures make use of semantic similarity in \textit{Gene Ontology}.
\end{mybox}

\begin{mybox}
    \textbf{Gene Ontology (GO)} - Whereas an \textit{ontology} is a formal representation of a body of knowledge within a given domain; a \textit{gene ontology} describes our knowledge of the biological domain w.r.t. \textit{molecular function}, \textit{cellular component} and \textit{biological process}. \cite{GOdoc}
\end{mybox}

\begin{mybox}
    \textbf{Gene Ontology \textit{annotations}} - A statement about the function of a particular gene. At the very minimum, a GO annotation consists of: \textit{gene product}, \textit{GO term}, \textit{reference} and \textit{evidence}. \cite{GOdoc}
\end{mybox}

\begin{mybox}
    \textbf{Gene Ontology \textit{term}} - A GO \textit{term} or class consists primarily of a definition, a label and a unique identifier. Whereas the ontology itself is a loosely hierarchical graph, the terms are nodes in this graph. \cite{GOdoc}
\end{mybox}

\begin{mybox}
    \textbf{Molecular Function (MF)} - Describes activities that occur at the molecular level (\textit{e.g., catalysis}). These activities are performed by an individual gene product \textit{or complexes}. Part of the process involves learning low dimensional vector embeddings for gene products and GO terms (\textit{using which the FS is computed}).  \cite{GOdoc}
\end{mybox}

\begin{mybox}
    \textbf{Sequence Homology} - Similarity due to shared ancestory, between DNA, RNA or protein sequences.
\end{mybox}

\begin{mybox}
    \textbf{Semantic Similarity (SS) (\textit{of GO terms})} - It is a means of comparing the similarity of GO terms based on ontology (graph) structure and annotation corpora.
\end{mybox}
