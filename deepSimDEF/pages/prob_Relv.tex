\begin{sloppypar*}
    In general, GO (\textit{Gene Ontology}) based FS (\textit{Functional Similarity})
    measures depend on slow FS computation and empirical SS (\textit{Semantic
    Similarity}) metric engineering. \hfill \break

    \begin{mybox}
        The following provides important context!
    \end{mybox}
    In literature, two computational classes of GO based FS measures are available,
    viz., \textbf{Ontology-based methods} and \textbf{Distributional-based methods}.
    The former utilizes either pair-wise \textit{Information Content} measures,
    which is computationally costly, or group (set) wise measures which are less
    compute intensive, but also, less accurate. In the latter, the similarity measures
    are based on the comparison of \textit{text definitions of a term} with that of its neighbours.
    Both of these classes rely on \textbf{manual metric} (\textit{e.g., MAX, BMA})
    \textbf{\& feature engineering} for \textit{aggregating GO terms' SS scores, prior to
    computing gene FS scores}.
\end{sloppypar*}