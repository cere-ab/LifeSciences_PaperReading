\begin{sloppypar*}

    Machine Learning based downstream tasks require that the genes, diseases etc.
    are represented numerically. \textbf{Representation Learning} converts complex
    data structures (free text, ontology based annotations) into numeric vectors
    (or, matrices and tensors) so that ML techniques may consume it. The complex
    data structures are built from various data sources, such as, \hfill\break

    \begin{tabularx}{\textwidth}{XXX}
        \textit{occurences in scientific literature} & \textit{CRISPR screens} & \textit{protein sequence} \\
        \textit{protein-protein interaction network} & \textit{gene expressions} & \textit{gene ontology annotation} \\
    \end{tabularx} \hfill\break
    \noindent Sampling Bias, a type of \textit{selection bias}, pertains to
    over-representation of certain members of a population in some experiment. 
    In research literature, some genes/diseases are more explored than others.
    Functional gene embeddings generated from such a literature corpus will contain
    this type of bias. This results in \textbf{under-performance} of the embeddings
    on downstream tasks for less studied genes/diseases. The authors have referred
    to this as \textbf{annotation inequality bias} caused by street-light-effect. \hfill\break

    \noindent In this study, the authors \textit{have not attempted to create any novel
    solution to the downstream tasks}, but rather, focused on studying the effect
    of various data sources on such tasks using off-the-shelf ML algorithms. They
    have, however, proposed an embedding generation scheme that they claim will
    remove the aforementioned bias.

\end{sloppypar*}