\begin{mybox}
    \textbf{Allele} - or \textit{Allelomorph}, is a variant of the sequence of
    nucleotides at some locus in a DNA molecule; a person inherits one allele from
    each parent for an autosomal gene
\end{mybox}

\begin{mybox}
    \textbf{Assays} - Some kind of test to determine content, function or qualities
\end{mybox}

\begin{mybox}
    \textbf{Autosomes} - Any of the numbered chromosomes other than the sex determining
    chromosomes
\end{mybox}

\begin{mybox}
    \textbf{Co-variates} - A co-variate affects the outcome of a response variable
    in a statistical trial along with the explanatory variables under consideration.
    A co-variate itself is not of direct interest
\end{mybox}

\begin{mybox}
    \textbf{Elastic regularisation} - Elastic Net is a regularized regression method
    that linearly combines the $L_1$ and $L_2$ penalities of LASSO and Ridge methods
\end{mybox}

\begin{mybox}
    \textbf{Exome} - A na\"{i}ve understanding of genome could be that it is made
    of alternating \textit{introns} and \textit{exons}. The set of all exons is the
    Exome. When a genome is transcribed, a particular cell type is formed which has
    exons drawn from the Exome
\end{mybox}

\begin{mybox}
    \textbf{Genotype} - The genotype of an organism is its complete set of
    genetic material
\end{mybox}

\begin{mybox}
    \textbf{Modality} - May refer either to different sources or types of input
    data
\end{mybox}

\begin{mybox}
    \textbf{Multi-collinearity} - A statistical concept where several independent
    variables in a model are correlated
\end{mybox}

\begin{mybox}
    \textbf{Phenotype} - The set of observable characteristics of an organism including
    morphology, developmental processes, biochemical, physiological as well as
    behavioral properties - generated from genotype and environmental factors
\end{mybox}

\begin{mybox}
    \textbf{Physiology} - branch of biology that studies how a living organism
    operates (functions and mechanisms)
\end{mybox}

\begin{mybox}
    \textbf{SNPs} - \textit{Single Nucleotide Polymorphisms}, is a germline (
    \textit{population of reproductive cells}) substitution of a single nucleotide
    at a locus in the genome
\end{mybox}

\begin{mybox}
    \textbf{Stratified} - When a population is heterogeneous with sub-groups of
    different sizes, instead of random sampling, the sub-groups are arranged in
    \textit{strata} and sampling is done proportionally to mitigate sampling bias
\end{mybox}

\begin{mybox}
    \textbf{Traits} - Various traits form the Phenotype set. Traits can be quantitative
    such as \textit{blood pressure}, \textit{height} or qualitative such as
    \textit{eye colour}. It characterises specific aspects of an organism
\end{mybox}

\begin{mybox}
    \textbf{Trait-gene} - May be a reference to a \textit{dominant allele}, gene
    responsible for the expression of a trait
\end{mybox}

\begin{mybox}
    \textbf{Z-score} - or, \textit{standard score}, is the number of standard
    deviations that a data point is above or below the mean of a measured quantity
\end{mybox}
